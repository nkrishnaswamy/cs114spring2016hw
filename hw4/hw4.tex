\def\DevnagVersion{2.13}

\documentclass[11pt,letterpaper]{article}
\usepackage{latexsym}
\usepackage{listings}
\usepackage[usenames,dvipsnames]{color}
\usepackage[pdftex]{graphicx}
\usepackage{url}
\usepackage{hyperref}
\usepackage{tabularx}

\begin{document}


\title{CS114 (Spring 2016) Homework 4 \\ Statistical Parser: CKY Algorithm\footnote{This assignment is adapted from Hal Daume's}}
\author{Due April 12, 2015}
\date{}
\maketitle

\section*{Introduction}

The starter code consists of several Python files, some of which you will need to read and understand in order to complete the assignment, and some of which you can ignore. 
All code is available from the github repo.

Files you will edit:
\begin{itemize}
\item \verb parser.py  The place where you'll put your code
\end{itemize}

Files you might want to look at:
\begin{itemize}

\item \verb grammar.py  The place where we put the grammar
\item \verb tree.py  The tree data structure we'll use. (same as nltk tree)
\item \verb util.py  A handful of useful utility functions.

\end{itemize}

Your code will be autograded for technical correctness. Please do not change the names of any provided functions or classes within the code, or you will wreak havoc on the grading script. 

\section{Probabilistic CKY algorithm}
For this problem, you will be  running the CKY algorithm by hand in 1.1 and implementing it in 1.2. If you think you already understand CKY, do 1.2 first and then use your code to do 1.1. 
If you are still shaky, do 1.1 first and 1.2 should be quite straightforward.

\subsection{By hand}
You are given a small grammar. Your task is to write down all of the possible subtrees for ``time flies like an arrow."  The CKY algorithm will help you do this efficiently, and it is really not painful to do by hand.   The grammar can be obtained from the starter code.
\begin{verbatim}
>>> from grammar import timeFliesPCFG
>>> print str(timeFliesPCFG)
Noun => flies	| 0.4
Noun => arrow	| 0.4
Noun => time	| 0.2
TOP => S	| 1.0
Det => an	| 1.0
VP => Verb PP	| 0.2
VP => Verb NP_PP	| 0.1
VP => Verb NP	| 0.6
VP => Verb	| 0.1
S => VP	| 0.2
S => VP PP	| 0.1
S => NP VP_PP	| 0.2
S => NP VP	| 0.5
VP_PP => VP PP	| 1.0
NP_PP => NP PP	| 1.0
Prep => like	| 1.0
PP => Prep NP	| 1.0
Verb => flies	| 0.5
Verb => time	| 0.3
Verb => like	| 0.2
NP => Noun	| 0.3
NP => Det Noun	| 0.7
\end{verbatim}

\begin{enumerate}
\item (Base case) Write down all of the possible subtrees and their probability that span over just one word. Don't forget the unary rules. In CKY terms, write down the subtrees and their probabilities in the cell (0,1), (1,2), (2,3), (3,4), and (4,5).
\item (Recursive case) Write down all of the possible subtrees and their probability that span over 2 words. In CKY terms, write down the subtrees and their probabilities in the cell (0,2), (1,3), (2,4), (3,5). 

For example, for cell (0,2), you will find the rules that can combine subtrees in (0,1) with the subtrees in (1,2), which you have derived in the previous step. 
Then apply all of those rules to generate bigger subtrees. 

\item (Recursive case) Write down all of the possible subtrees and their probability that span over 3 words. In CKY terms, write down the subtrees and their probabilities in the cell (0,3), (1,4), (2,5). 

For example, for cell (1,4), you will find the rules that can combine subtrees from (1, 2) and (2, 4) and the rules that can combine subtrees from (1, 3) and (3, 4).
Then apply all of those rules to generate bigger subtrees. 

\item (Recursive case) In CKY terms, write down the subtrees (spanning 4 words) and their probabilities in the cell (0,4), (1,5)

\item (Termination) Write down the trees (and their probabilities) that span the entire sentence i.e. apply rules that can combine (0,1) \& (1,5), (0,2) \& (2,5), (0,3) \& (3,5), and (0,4) \& (4,5). The complete tree must have \verb TOP  as the root.

\end{enumerate}

\subsection{By machine}
You have a nearly-complete implementation of CKY in \verb parser.py .  You only have to fill in a blank. 
The provided implementation initializes the bottom cells of the chart, then applies unary rules there, and applies binary rules in the entire chart. All is left for you to do is apply unary rules in the recursive case, but that requires you to understand the whole algorithm and the data structures involved. (Hint: the solution only needs around 7 lines of code, so it is not crazy.) If you've correctly implemented this, and loaded all the relevant files, you should be able to run:

\begin{verbatim}
>>> print timeFliesSent
['time', 'flies', 'like', 'an', 'arrow']

>>> parse(timeFliesPCFG, timeFliesSent)
(TOP: (S: (NP: (Noun: 'time')) (VP: (Verb: 'flies') 
              (PP: (Prep: 'like') (NP: (Det: 'an') (Noun: 'arrow'))))))

\end{verbatim}

Note that you cannot get this output without correctly handling unary rules, because you'll never be able to get the {\verb TOP \verb => \verb S  at the top.


\section*{Submission}
Submit your solution from 1.1 and your version of \verb parser.py  on Latte. If you draw trees by hand for 1.1, scan the solutions and convert into pdf. 
\end{document}