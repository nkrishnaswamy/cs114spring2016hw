\def\DevnagVersion{2.13}% Assignment 2 for CMSC723
% Finite State Machines

\documentclass[11pt,letterpaper]{article}
\usepackage{latexsym}
\usepackage{listings}
\usepackage[usenames,dvipsnames]{color}
\usepackage[pdftex]{graphicx}
\usepackage{url}
\usepackage{hyperref}


%@modernhindi

% \date{}
\begin{document}


\title{CS114 (Spring 2016) Homework 1\\Finite State Transducers}
\author{Due February 5, 2016}
\date{}
\maketitle

% \maketitle
\lstset{stringstyle=\ttfamily,language=Python,showstringspaces=False,tabsize=8,frameround=tttt,
		,keywordstyle=\color{Orange}\bfseries, stringstyle=\ttfamily\color{Green}
		,columns=fullflexible,identifierstyle=\ttfamily
		% , commentstyle=\itshape\color{Red}
}

This assignment is mostly on finite state automata and will require you to
create automata in Python.
There are a total of three problems, and we
have provided some code to get you started, as well as a few utilities
that will make it easier for you to debug and test your code.  

Some minimal test cases are provided as well. 
Before you turn in your work, you should at least pass all of the test cases we provided although it is neither sufficient nor necessary for getting full credit. 
You should add your own test cases to help yourself debug your program.

\section*{Problem 1: Regular Expressions }
\begin{enumerate}
\item Implement a function that determines whether a given string is a valid email address or not. (\verb is_email_address()  in \verb recognizer.py )
\item Implement a function that determines whether a given string is a valid phone number or not e.g (123)-456-7890 or 123-456-7890 or 123-4567890 or other valid patterns. 
(\verb is_phone_number()  in \verb recognizer.py )
\end{enumerate}

\section*{Problem 2: Soundex\footnote{ 
This problem is adapted from Jordan Boyd-Graber's course at UMD.} } % (fold)
\label{sec:problem_2}
The Soundex algorithm is a phonetic algorithm
commonly used by libraries and the Census Bureau to represent people's
names as they are pronounced in English. It has the advantage that
name variations with minor spelling differences will map to the same
representation, as long as they have the same pronunciation in
English. Here is how the algorithm works:
\begin{description}
	\item[Step 1:] Retain the first letter of the name. This may
          be uppercased or lowercased.
	\item[Step 2:] Remove all \textbf{non-initial} occurrences of
          the following letters: \texttt{a, e, h, i, o, u, w, y}. (To
          clarify, this step removes all occurrences of the given
          characters \emph{except} when they occur in the first
          position.)
	\item[Step 3:] Replace the remaining letters (except the
          first) with numbers:
	\begin{itemize}
		\item \texttt{b, f, p, v} $\rightarrow 1$
		\item \texttt{c, g, j, k, q, s, x, z} $\rightarrow 2$
		\item \texttt{d, t} $\rightarrow 3$
		\item \texttt{l} $\rightarrow 4$
		\item \texttt{m, n} $\rightarrow 5$
		\item \texttt{r} $\rightarrow 6$
	\end{itemize}
 	If two or more letters from the same number group were
        adjacent in the \emph{original} name, then \emph{only} replace
        the first of those letters with the corresponding number and
        ignore the others.
 	\item[Step 4:] If there are more than $3$ digits in the
          resulting output, then drop the extra ones.
 	\item[Step 5:] If there are less than $3$ digits, then pad at
          the end with the required number of trailing zeros.
\end{description}


 The final output of applying Soundex algorithm to any input string
 should be of the form \texttt{Letter Digit Digit
   Digit}. Table~\ref{tbl:soundex} shows the output of the Soundex
 algorithm for some example names. 

\begin{table}[htbp]
	\begin{center}
		\begin{tabular}{|c|c|}
			\hline
			\textbf{Input} & \textbf{Output} \\ \hline \hline
			Jurafsky & J$612$ \\ \hline
			Jarovski & J$612$ \\ \hline
			Resnik & R$252$ \\ \hline
			Reznick & R$252$ \\ \hline
			Euler & E$460$ \\ \hline
			Peterson & P$362$ \\ \hline
			\end{tabular}
		\caption{Example outputs for the Soundex algorithm.}\label{tbl:soundex}
\end{center}
\end{table}

 Construct an \textsc{fst} that implements the Soundex
 algorithm. Obviously, it is non-trivial to implement a single
 transducer for the entire algorithm. Therefore, the strategy we will
 adopt is a bottom-up one: implement multiple transducers, each
 performing a simpler task, and then compose them together to get the
 final output. One possibility is to partition the algorithm across
 three transducers:
\begin{enumerate}
	\item \textbf{Transducer 1}: (\verb letters_to_numbers ) Performs steps $1$-$3$ of the
          algorithm, i.e, retaining the first letter, removing letters
          and replacing letters with numbers.
	\item \textbf{Transducer 2}: (\verb truncate_to_three_digits ) Performs step $4$ of the
          algorithm, i.e., truncating extra digits.
	\item \textbf{Transducer 3}: (\verb add_zero_padding ) Performs step $5$ of the
          algorithm, i.e., padding with zeros if required. 
\end{enumerate}
 Note that each of these three transducers will have characters as
 input/output symbols.  

 To make things easier for you, we have provided the file
 \texttt{soundex.py} which is where you will write your code. It
 already imports all needed modules and functions (including
 \texttt{fsmutils.py}). It also creates three transducer objects---as
 dictated by the bottom-up strategy outlined above---such that all you
 should have to do is to figure out the states and arcs required by
 each transducer. It also contains code that allows you to input a
 single name on the command line to get the output.

Note that while we have provided you with sample unit
tests containing some names, it might be very useful to test
your code on other names. For comparison purposes, you may
use one of the many Soundex calculators available online to create more test cases.

\section*{Problem 3: English Morphology}

English often requires some spelling changes at the morpheme boundary between the stem and the affixes that indicate inflection verb.
Specifically, we will build a finite-state transducer to handle the K-insertion rule in English.

If a verb ends with vowel + c, then we add k before adding in the -ed and -ing suffixes. For example \\

\noindent panicked $\leftrightarrow$ panic + ed $\leftrightarrow$ panic+past form\\
panicking $\leftrightarrow$ panic + ing $\leftrightarrow$ panic+present participle form \\
lick $\leftrightarrow$ lick + ed $\leftrightarrow$ lick+past form \\
lick $\leftrightarrow$ lick + ing $\leftrightarrow$ lick+participle form \\

As you can see in the examples, the morphological parsing process can be done in two steps (although you may skip the intermediate step). So you might need to build two separate FSTs and compose them. 
The lexicon FST only has to include two inflectional morphemes (-ed and -ing) and five verbs: want, sync, panic, havoc, and lick. The other FST will handle the actual K-insertion rule. 
This method is very similar to the example given in the chapter.

The final FST will be used as a morphological parser to transduce the surface form (e.g panicked) to the lexical form (panic + past form). 
The inverted FST will be used to transduce the lexical form to the surface form. 




\section*{Turning in Your Assignment}

Submit your completed code to Latte. We are using a grading script to grade the work, so please make sure:
\begin{itemize}
  \item You submit each file individually. Do not zip or tar the files. 
  \item You do not change filenames, function names, or the \textsc{api}.
  \item Do not add \texttt{print} statements to the code you turn in
  \item Add your name as a comment to all files you turn in
\end{itemize}

\newpage
\section*{How to use the FST code}

We've provided a modified \textsc{fst} implementation from \textsc{nltk}.
However, the documentation is less than perfect. Therefore, in this
section, we will give you a brief introduction to everything that you
will need to understand how to build finite state transducers in Python.

To start building \textsc{fst}s, you need to first import the
\texttt{fst} module into your program's namespace. Then, you need to
instantiate an \texttt{FST} object. Once you have such an object, you
can start adding states and arcs to it. Listing~\ref{example} shows
how to build a very simple finite state transducer---one that removes
all vowels from any given word.

\begin{lstlisting}[float, label=example,caption=A 1-state transducer that deletes vowels, frame=trBL,escapechar=*, numbers=left, numberstyle=\tiny, numberblanklines=false]
# import the fst module
import fst

# import the string module
import string

# Define a list of all vowels for convenience
vowels = *\texttt{[}*'a', 'e', 'i', 'o', 'u'*\texttt{]}*

# Instantiate an FST object with some name
f = fst.FST('devowelizer')

# All we need is a single state ...
f.add_state('1')

# and this same state is the initial and the final state
f.initial_state = '1'
f.set_final('1')

# Now, we need to add an arc for each letter; if the letter is a vowel
# then then transition outputs nothing but otherwise it outputs the same
# letter that it consumed.
for letter in string.ascii_lowercase:
*\qquad*if letter in vowels:
*\qquad**\qquad*f.add_arc('1', '1', letter, '')
*\qquad*else:
*\qquad**\qquad*f.add_arc('1', '1', letter, letter)

# Evaluate it on some example words
print ''.join(f.transduce(*\texttt{[}*'v', 'o', 'w', 'e', 'l'*\texttt{]}*))
print ''.join(f.transduce('e x c e p t i o n'.split()))
print ''.join(f.transduce('c o n s o n a n t'.split()))

\end{lstlisting}

 Feel free to try out the example to see how it works on some of your
own input. There are a few points worth mentioning:
\begin{enumerate}
	\item The Python \texttt{string} module comes with a few
          built-in strings that you might be able to use in this
          assignment for purposes of iteration as used in the example
          on line $23$. These are:
	\begin{itemize}
		\item \texttt{string.letters} : All letters, upppercased and lowercased
		\item \texttt{string.ascii\_lowercase} : All lowercased letters
		\item \texttt{string.ascii\_uppercase} : All uppercased letters
	\end{itemize}
	\item States can be added to an FST object by using its
          \texttt{add\_state()} method. This method takes a single
          argument: a unique string identifier for the state. Our
          example has only one state (line $13$). Furthermore, there
          can only be \textbf{one} initial state and this is indicated
          by assigning the state identifier to the FST object's
          \texttt{initial\_state} field (line $17$). However, there
          may be multiple final states in an FST. In fact, it is
          almost always necessary to have multiple final states when
          working with transducers. All final states may be so
          indicated by using the FST object's \texttt{set\_final()}
          method (line $18$).
	\item Arcs can be added between the states of an FST object by
          using its \texttt{add\_arc()} method. This method takes the
          following arguments (in order): the starting state, the
          ending state, the input symbol and, finally, the output
          symbol. 
	
	 \item An FST object can be evaluated against any input string
           by using its \texttt{transduce()} method. Here's how:
	\begin{enumerate}
		\item If your transducer uses \textbf{characters} as
                  input/output symbols, then the input to
                  \texttt{transduce()} must be a \textbf{list of
                    characters}. You may either directly input a list
                  of characters (line $30$) or you may convert a
                  string to a list of characters by spacing out its
                  characters and calling its \texttt{split()} method
                  (lines $31$ and $32$).
		\item If your transducer uses \textbf{words} as
                  input/output symbols, then the input to
                  \texttt{transduce()} should be a \textbf{list of
                    words}. Again, you can either explicitly use a
                  list of words or call the split method on a string
                  of words separated by whitespace. For example, say
                  your FST maps from strings like \emph{ten} and
                  \emph{twenty} to number strings \emph{10} and
                  \emph{20}, then to evaluate it on the input string
                  \emph{ten twenty}, you should use either:
		\begin{lstlisting}[label=foo2, frame=trBL,escapechar=*]
		f.transduce('ten twenty'.split())
		\end{lstlisting}
		\begin{center}{OR}\end{center}
		\begin{lstlisting}[label=foo3, frame=trBL,escapechar=*]
		f.transduce(*\texttt{[}*'ten', 'twenty'*\texttt{]}*)
		\end{lstlisting}
	\end{enumerate}
	\item In \verb fsmutils.py  , we provide the code for composition. 
	It does not actually perform the composition and return a newly composed FST. It cascades the output from one FST to the next. 
	\begin{lstlisting}[label=composechars, frame=trBL,escapechar=*]
	from fsmutils import compose
	output = compose(S, f1, f2, f3)
	\end{lstlisting}
		The above function call computes (\texttt{f3} $\circ$
        \texttt{f2} $\circ$ \texttt{f1})(\texttt{S}). i.e., it will
        first apply transducer \texttt{f1} to the given input
        \texttt{S}, use the output of this transduction as input to
        transducer \texttt{f2} and so on and so forth. \texttt{S} must be
        a list of characters.         
\end{enumerate}

\end{document}
